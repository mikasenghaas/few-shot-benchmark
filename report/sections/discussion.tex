\section{Discussion}

% General Benchmark
\textbf{General Benchmark}. The average performance on both datasets show that the all methods are capable of learning from limited samples
in a biomedical setting. The higher overall performance of the methods on \texttt{TM} 
suggests that \texttt{SP} is a more challenging dataset. This could be due to the quality of extracted 
embeddings, and possibly the lack of overlap between the train and test sets, in contrast to \texttt{TM}.


For the \texttt{SP} dataset, we observe that integration of the SOT module improves the performance of all methods
except from \texttt{B} for which the performance in fact decreases by almost 20\%. Such dramatic decrease in performance
could be due to the fact that \texttt{B} is a relatively simple model and the SOT module introduces additional complexity
that the model is not able to handle. Interestingly, the performance of \texttt{B++} increases by almost 17\% with the SOT module,
which is the largest increase among all methods. This is suprising since the two methods differ in a very few small details.
Namely, \texttt{B++} normlizes the embeddings before the final classification layer whose weights are also normalized. Furthermore,
\texttt{B++} increases the temperature of the softmax function to make the model more confident in its predictions.

For the \texttt{TM} dataset, we observe that the SOT module has a minimal impact on the performance of all methods
except from \texttt{MT} for which the performance increases by 5\%. A possible explanation for this is that the
average performance of the methods on \texttt{TM} is already very high, and thus the SOT module has little room
for improvement. 

% Way-Shot Analysis
The way-shot analysis results for \texttt{MT} on the \texttt{TM} dataset confirm that the SOT module keeps
improving the performance of the model as the number of classes increases, and the number of samples per class decreases.
Further, we can see that despite making the configuration more challenging, \texttt{MT} is able to 
maintain a high test accuracy of arouund 80 \%. This is a very promising result, since it shows that the model
is able to learn from very few samples, and thus could be used in a real-world setting where the number of samples
is limited.

% SOT hyperparameters ablation