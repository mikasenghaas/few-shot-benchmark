\section{Discussion}

% General Benchmark
The collective analysis of average performance across both datasets underscores the competence 
of all methods in extracting knowledge from biomedical datasets with limited number of samples per class. 
Notably, the overall great performance on \texttt{TM} implies that \texttt{SP} poses a more difficult challenge. 
This discrepancy may stem from variations in the quality of extracted embeddings and potential disparities in the train-test set overlap, 
particularly when compared to the \texttt{TM} dataset.

Examining the \texttt{SP} dataset, an intriguing pattern emerges: the incorporation of the SOT module 
enhances the performance of all methods except for \texttt{B}, where performance sees a notable decline of almost 20\%. 
This stark drop may be attributed to the inherent simplicity of model \texttt{B}, struggling to grapple with the added complexity introduced by the SOT module. 
Remarkably, \texttt{B++} experiences a substantial 17\% performance boost with the SOT module, marking the most significant improvement among all methods. 
This result is surprising given the minimal distinctions between the two methods. 
Notably, \texttt{B++} normalizes embeddings before the final classification layer, with normalized weights. 
Additionally, it elevates the temperature of the softmax function, fostering greater confidence in model predictions.

Turning attention to the \texttt{TM} dataset, the impact of the SOT module appears marginal for all methods except \texttt{MT}, 
where performance sees a modest 5\% increase. Plausible reasoning suggests that, on \texttt{TM}, 
methods already achieve a high average performance, limiting the potential for substantial enhancement through the SOT module.

% Way-Shot Analysis
The way-shot analysis demonstrates clear trends: With increasing number of classes to distinguish, performance decreases from approximately 95\% with 2 classes to around 80\% with 10 classes. Notably, the challenge intensifies, widening the confidence interval range. This underscores the increasing importance of support vectors as class count rises.

Examining the second plot of Figure \ref{fig:way-shot}, an intriguing observation emerges:  possessing more than 5 support vectors doesn't necessarily enhance the model's performance.  Importantly, the figure indicates that, across varying class and support vector counts, the \texttt{MN} with SOT module tends to exhibit marginally lower performance on average compared  to its counterpart without the module. 