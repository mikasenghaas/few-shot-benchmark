\documentclass{article}

% Recommended, but optional, packages for figures and better typesetting:
\usepackage{microtype}
\usepackage{graphicx}
\usepackage{subfigure}
\usepackage{booktabs} % for professional tables
\usepackage{hyperref}
\usepackage{multirow}
\usepackage{multicol}
\usepackage{amsfonts}
\usepackage{hyperref}
\usepackage{float}
\usepackage[table]{xcolor}

\usepackage[accepted]{cs502}


% \icmltitlerunning{Submission and Formatting Instructions for ICML 2021}

\begin{document}

\twocolumn[
    \icmltitle{
        Benchmarking Few-Shot Learning in Biomedicine:\\
        Insights from Cell Classification and Protein Function Prediction
    }

    \begin{icmlauthorlist}
    \icmlauthor{Mika Senghaas}{}
    \icmlauthor{Ludek Cizinsky}{}
    \icmlauthor{Adam Barla}{}
    \end{icmlauthorlist}

    \vskip 0.3in
]

\begin{abstract}
    Learning from few samples remains a major challenge in machine learning, especially in biomedical tasks were data availability is often a limiting factor. Our study conducts a comprehensive evaluation of prominent few-shot learning algorithms, as cited in~\cite{baseline, matchingnet, protonet, maml} applied to two biomedical tasks  adapted for few-shot learning: single-cell type prediction using the Tabula Muris dataset~\cite{tabula2018} and protein function prediction with SwissProt data~\cite{uniprot2019}.
    The results demonstrate the robustness of these algorithms in learning effective representations even from limited data. Notably, employing the transductive SOT~\cite{sot} feature transform combined with meta-learners emerged as a highly effective approach, facilitating rapid and precise knowledge transfer.
\end{abstract}
\section{Introduction}

Biomedical research often grapples with data scarcity, typically due to high costs of data collection and expert-dependent annotation processes. 
Traditional machine learning approaches often fall short in these data-limited settings as they require extensive training samples and iterations. 
Few-shot learning algorithms, tailored to discern distinct features from minimal data, offer a promising alternative. 
This study aims to assess the applicability of such methods in the biomedical domain. We examine four established few-shot learning techniques — Baseline(++)~\cite{baseline}, 
Matching Networks~\cite{matchingnet}, Prototypical Networks~\cite{protonet}, and Model-Agnostic Meta-Learning (MAML)~\cite{maml} — across two distinct biomedical tasks. 
The first involves predicting cell types based on gene expression, using the Tabula Muris dataset~\cite{tabula2018}. The second task focuses on predicting protein functions 
from their sequence embeddings, using the SwissProt dataset~\cite{uniprot2019}.

All of the above approaches rely on meaningful embeddings of features. However, a notable challenge arises due to the potential discrepancy in data 
distributions between samples seen during meta-training and samples of novel classes seen during meta-testing. This discrepancy can result in embeddings that are 
not fully transferable, leading to suboptimal performance in downstream tasks. Self-Optimal-Transport (SOT)~\cite{sot} is a feature transform module, 
grounded in probabilistic interpretations, that aims to mitigate this issue. We include the SOT feature transform module in all of the above mentioned few-shot learning methods to 
study its effectiveness in the biomedical domain.

In summary, our study presents three primary contributions. First, we train and assess leading few-shot learning algorithms on two unique biomedical tasks. We 
report the performance of these algorithms on both the original and SOT-transformed features. Second, we explore the relationship between method's performance and
the number of training samples per class as well as the number of classes. Third, we conduct an ablation on SOT's hyperparameters to understand their impact on
performance. We hope that our study will serve as a benchmark for future research in the biomedical domain.


\textit{All code and data to reproduce the experiments is available on \href{https://github.com/mikasenghaas/few-shot-benchmark}{GitHub} and the reported experiments can be accessed on \href{https://wandb.ai/metameta-learners/few-shot-benchmark}{W\&B}.}

\section{Dataset}

% Introduce datasets
% This project utilises two biomedical datasets adapted for few-shot learning. 

% TODO: (Optional) Figure about difficulty of the task by showing PCA features of samples within k episodes (visually explore the class separation)

% Tabula Muris
The first dataset, \textbf{Tabula Muris}~\cite{tabula2018} (denoted as \texttt{TM}), comprises over 100,000 mouse cells' gene expression 
data and annotations about the cell ontology class (cell type). The task is to predict the cell type based on the gene expression data. 
To address the sparsity and skewed distribution of raw gene expressions, preprocessing included gene and cell filtering, log-transformation, 
and mean normalization, with zero imputation. Post-processing, the dataset features 105,960 cells across 125 cell types.

For the few-shot learning task, the focus is on generalisation across different tissues. The dataset is divided into training, validation, and testing splits, 
each representing distinct tissue types: 15 for training and four each for validation and testing. Despite some overlap in cell types across tissues this structure 
ensures diverse tissue representation and makes the task of cell type prediction in cells from novel tissues challenging.

% SwissProt
The second dataset, \textbf{SwissProt}~\cite{uniprot2019} (denoted as \texttt{SP}), is an extensively annotated protein sequence database 
featuring 14,251 sequences, enriched with comprehensive information on their functions, structures, and biological roles. 
This project utilises pre-computed sequence embeddings obtained from ESM-2~\cite{esm-2}, a state-of-the-art protein language model, as input data. 
The goal is to predict proteins' functions. In total, there are 884 unique annotated protein functions. 
The dataset is divided into three splits with no overlap in the targets.

% TODO: Get more clarity on gene ontology (GO) annotations
% TODO: Understand why the classes that we get don't match the ones in the project presentation

\begin{table}[h]
    \centering
    \caption{\textbf{Dataset Statistics.} The table shows the number of samples and targets in each split of the two datasets.}
    \vspace{5pt}
    \label{tab:data}
    \begin{tabular}{lllll}
        \toprule
        & \textbf{Split} & \textbf{\#Samples} (\%) & \textbf{\#Targets} & \textbf{Overlap} \\
        \midrule
        \multirow{3}{*}{\texttt{TM}} 
        & Train. & 65,846 (62\%) & 59 & N/A \\
        & Val. & 15,031 (14\%) & 47 & 10 \\
        & Test. & 25,083 (24\%) & 37 & 4 \\
        \\
        \multirow{3}{*}{\texttt{SP}} 
        & Train. & 12,141 (85\%) & 636 & N/A \\
        & Val. & 1,407 (10\%) & 159 & 0 \\
        & Test. & 703 (5\%) & 89 & 0 \\
        \bottomrule
    \end{tabular}
\end{table}

\begin{figure*}[!]
    \centering
    \includegraphics[width=0.9\linewidth]{./figures/benchmark-method-perf.pdf}
    \caption{\textbf{Benchmark Results.} Test accuracy of all methods on \texttt{TM} (left) and \texttt{SP} (right) in the 5-way-5-shot setting. The plot shows the mean accuracy over 600 episodes and the 95\% confidence interval.}
    \label{fig:benchmark-perf}
\end{figure*}

\section{Methods}

Few-shot learning algorithms can be classified into two main categories: \textit{Transfer learning} involves a two-phase process of pre-training on a large dataset to learn general data representation, 
followed by fine-tuning on the target task with limited data. \textit{Meta-learning}, conversely, leverages past experiences from a series of related tasks to efficiently tackle a new task with sparse data. 
These algorithms undergo meta-training, where the model encounters various tasks, mimicking the target few-shot learning scenario. Each task comprises randomly chosen support and query samples from identical class sets, 
training the model to adapt to support samples and classify query samples.

In the following we describe the high-level idea of all methods considered within this study. For more details on the methods please refer to the original papers.

The \textbf{Baseline}~\cite{baseline} model implements the fine-tuning paradigm. During meta-testing, the model is fine-tuned on support samples and then classifies query samples. Within this study, we consider two variants of the trainable classification head - one learns a traditional linear and the other a cosine similarity layer. We refer to these as Baseline (\texttt{B}) and Baseline++ (\texttt{B++}), respectively.

\textbf{Prototypical Networks}~\cite{protonet} (\texttt{PN}) learn an embedding space that clusters samples from the same class close together. 
Query samples are then classified according to the distance to the average support sample (prototype) of each class.

\textbf{Matching Networks}~\cite{matchingnet} (\texttt{MN}) are similar to \texttt{PN}. However, in \texttt{MN}, the distance between a query sample is computed to all support samples and then aggregated. 
Importantly, before the distance computation, \texttt{MN} contextualises both support and query samples by re-embedding them using an LSTM.

Finally, \textbf{Model Agnostic Meta Learning}~\cite{maml} (\texttt{MAML}) is an optimisation-based meta-learning approach that aims to learn an effective weight initialisation that can be adapted to new tasks in a small number of gradient steps.

% TODO: Mention problems with reduced dimensionality of SOT embeddings.

The \textbf{Self-Optimal Transport} (SOT)~\cite{sot} feature transform is a parameterless and fully differentiable method for transforming feature vectors. SOT embeddings are notable for their interpretability and potential to upgrade a set of features to facilitate  downstream matching or grouping related tasks, as encountered frequently in few-shot learning settings.

SOT fundamentally utilises Optimal Transport (OT) on a square distance matrix (e.g. cosine similarity matrix) of input features, 
leading to embeddings that reflect the \textit{direct} similarity and \textit{third-party} agreement of samples to each other. 
Mathematically, SOT is a function \(T: \mathbb{R}^{n \times d} \rightarrow \mathbb{R}^{n \times n}\) that maps \(n\) samples in \(d\)-dimensions to a 
re-embedded SOT vectors in \(n\)-dimensions. The SOT embeddings are computed from an iterative optimisation algorithm known as the Sinkhorn-Knopp 
algorithm~\cite{sinkhorn-knopp} that solves a regularised version of the OT problem.

In few-shot learning contexts, SOT helps align independently embedded support and query samples by jointly embedding them according to their similarities to each other - an example of \textit{transductivity}. 
The SOT feature transform is used in state-of-the-art methods in common few-shot learning benchmarks~\cite{sot}. Within our study we employ the SOT feature transform module on the embeddings obtained from the backbone network. Critically, we shuffle the query samples before the forward-pass to avoid learning a trivial mapping from sample position to class label.
\section{Methodology}

An experiment in our study is defined as a combination of a few-shot learning \textit{method}, optionally including the \textit{SOT} feature transform, trained and evaluated on a \textit{dataset} within 
a specified few-shot learning setting, characterised by the number of classes (\textit{n-way}) and the number of samples per class (\textit{n-shot}).

\subsection{Experiment Setup}

% Backbone
\textbf{Backbone.} All experiments employ a fully-connected feed-forward neural network with batch normalisation, ReLU activation, and dropout. 
The network has two hidden layers, with hidden dimensions being tuned for each experiment.

% Model training
\textbf{Training.} Training of the models is conducted for a maximum of 40 epochs, employing the Adam optimiser with varying learning rates. 
We implement early stopping after five epochs of no improvement in validation accuracy. 

% Hyperparameter tuning
\textbf{Tuning.} Hyperparameter tuning is performed for all models that include the SOT module, unless specified otherwise. Tuning includes the learning rate ($\lambda = \{0.1, 0.01, 0.001\}$) for all methods as well as the backbone's hidden dimension size ($\kappa = \{64, 512, 1024\}$). For models including the SOT module, we adapt the hyperparameter grid of \citeauthor{sot}, namely the regularisation parameter ($\gamma = \{1.0, 0.1, 0.01\}$) and the choice of distance metric ($\delta = \{cosine, euclidean\}$). 
The model demonstrating the best performance on the validation split is evaluated on the test split and reported.

% Evaluation
\textbf{Evaluation.} A model performance's is reported through the mean and 95\% confidence interval of the few-shot accuracy, calculated over 600 episodes with each episode utilising five query samples per class.

\subsection{Experiments}

Due to the impracticality of exhaustive hyperparameter grid searching across all experimental configurations, 
we have structured our experiments into two distinct groups. Each group fixes certain parameters, allowing us to focus on the impact of the variables of interest.

\textbf{General Benchmark.} This experiment group evaluates models, with and without SOT feature transform, on both datasets in a 5-way-5-shot setting, 
comprising 20 experiments. The aim is to analyse the influence of the method, dataset, and SOT module on few-shot learning performance.

\textbf{Way-Shot Analysis.} In the second group we investigate the performance in various few-shot learning settings, exploring combinations of n-way ({2, 4, 6, 8, 10}) and n-shot ({1, 5, 10, 15, 20}). Here, we fix the method to \texttt{MN} and the dataset to \texttt{TM}, resulting in 50 experiments. For this experiment, we don't fine-tune the hyperparameters but instead use the best-performing hyperparameters from the general benchmark experiment.
\section{Results}


\textbf{General Benchmark.} Table \ref{tab:tuned-benchmark} shows the experiment results. Without the SOT module, models average 87\% accuracy on the \texttt{TM} dataset and 66\% on the \texttt{SP} dataset, demonstrating their ability to learn from limited samples. \texttt{MAML} and \texttt{B} are the top performers across datasets, whereas \texttt{B++} performs worst out of all methods.

With the SOT module the average accuracy on both datasets increases to 93\% on \texttt{TM} and 83\% on \texttt{SP}. This enhancement, however, is not uniform. Meta-learners significantly benefit, with their performance jumping to 99\% accuracy, showing an average increase of 12 percentage points on \texttt{TM} and 48 percentage points on \texttt{SP}. In contrast, non-meta-learners do not gain from the SOT module, with \texttt{B}'s performance declining and \texttt{B++}'s remaining static.


\begin{table}[h]
\caption{Results of the benchmark experiment.}
\label{tab:tuned-benchmark}
\centering
\begin{tabular}{llllr}
\toprule
 &  & Acc & w/ SOT & Diff (\%) \\
\midrule
\multirow[c]{5}{*}{TM} & B & $90.7 \pm 0.7$ & $86.3 \pm 0.9$ & -4.8 \\
 & B++ & $81.9 \pm 0.9$ & $82.8 \pm 0.9$ & \bfseries 1.1 \\
 & MAML & $92.8 \pm 0.5$ & $99.2 \pm 0.1$ & \bfseries 6.9 \\
 & MT & $84.6 \pm 0.8$ & $99.7 \pm 0.1$ & \bfseries 17.9 \\
 & PT & $87.1 \pm 0.8$ & $98.6 \pm 0.2$ & \bfseries 13.2 \\
\midrule
\multirow[c]{5}{*}{SP} & B & $69.2 \pm 0.7$ & $55.7 \pm 0.8$ & -19.5 \\
 & B++ & $64.1 \pm 0.7$ & $64.6 \pm 0.7$ & \bfseries 0.8 \\
 & MAML & $68.7 \pm 0.7$ & $98.0 \pm 0.2$ & \bfseries 42.6 \\
 & MT & $68.2 \pm 0.8$ & $99.8 \pm 0.1$ & \bfseries 46.3 \\
 & PT & $63.5 \pm 0.7$ & $99.2 \pm 0.1$ & \bfseries 56.2 \\
\bottomrule
\end{tabular}
\end{table}
% \caption{
%     \textbf{Benchmark Results.} Test accuracy of all methods on \texttt{TM} and \texttt{SP} in the 5-way-5-shot setting. We depict the average accuracy and the 95\% confidence interval both without (left) and with SOT (right) and the difference.
%     \vspace{5pt}
% }


\textbf{Way-Shot Analysis.} Figure \ref{fig:way-shot} presents the way-shot analysis results for \texttt{PN} on the \texttt{TM} dataset, with and without the SOT module. The left subplot shows test accuracy against the number of classes, and the right subplot against the number of samples per class.

Without the SOT module, there's a predictable trend: accuracy decreases with more classes but increases with additional samples per class. However, this pattern changes with the SOT module. Here, accuracy remains stable regardless of the number of ways or shots. Remarkably, in challenging scenarios like 10-way-1-shot, the model sustains a high test accuracy of around 97\%.

\begin{figure}[h!]
    \centering
    \includegraphics[width=1\columnwidth]{../figures/way-shot.pdf}
    \caption{\textbf{Way-Shot Analysis.} Test accuracy of \texttt{PN} on the \texttt{TM} dataset with and without the SOT module in various few-shot learning settings for fixed n-way (left) and n-shot (right). Individual points represent a single experiment. We show the regression line with a 95\% confidence interval.}
    \label{fig:way-shot}
\end{figure}


\textbf{SOT Interaction.} Figure~\ref{fig:sot-interaction-scatter} shows the interaction effects between the SOT module and the subsequent embedding modules (\texttt{SE} and \texttt{QE}) in \texttt{MN}, revealing notable patterns.

Each component boosts performance on its own. The model's performance falls to 40\% without additional encoding of embeddings from the backbone. Introducing either \texttt{SE}, \texttt{QE} or SOT elevates performance to around 60\%, matching the \texttt{PN} model's performance without the SOT module. Most importantly, the significant performance jump seen in Experiment 1 occurs only when the SOT module is paired with at least one subsequent embedding module. This indicates a non-linear, critical interaction between the SOT module and the embedding modules, essential for the enhanced results observed in Experiment 1.

\begin{figure}[h!]
    \centering
    \includegraphics[width=0.75\columnwidth]{../figures/sot-interaction-scatter.pdf}
    \caption{\textbf{SOT Interaction Ablation.} Test accuracy of \texttt{MN} with and without the SOT module in various configurations on the \texttt{TM} dataset in the 5-way-5-shot setting. The inclusion of \texttt{SE}, \texttt{QE} and \texttt{SOT} are binary variables that are encoded on the axis and through colour. Test accuracy is encoded as the size of the marker.}
    \label{fig:sot-interaction-scatter}
\end{figure}


\section{Discussion}

% Finding 1: Robustness of few-shot learning methods on biomedical tasks
Our benchmarks reveal that when data is meticulously pre-processed, various few-shot learning algorithms excel in the biomedical field. This outcome is encouraging, showing these methods' robustness in managing the intricate and specialised nature of biomedical data.

% Finding 2: SOT improves meta-learners but not non-meta learners
The most notable finding of our study is the significant role of the SOT module in enhancing the performance of meta-learners in challenging few-shot learning scenarios, as seen on both datasets. This leads to near-perfect generalisation. However, this performance boost is absent in non-meta learners, specifically \texttt{B} and \texttt{B++}. This difference likely stems from their distinct training methods. Meta-learners undergo episodic training that mirrors a few-shot environment with support and query samples across classes. The integration of the transductive SOT feature transform module enables simultaneous optimisation of the main network and the SOT module. This enhances the alignment between support and query samples, crucial for classifying query samples. Conversely, non-meta learners train with random mini-batches and lack a concept of support or query samples. Their inferences are independent, making the SOT's role in facilitating interaction between support and query samples irrelevant. For instance, in Figure~\ref{fig:sot-embeddings}, the SOT embeddings in the \texttt{PN} model demonstrate clear class-based clustering for both support and query samples, a feature not observed in the \texttt{B} model.

% TODO: Adjust title of right subplot
\begin{figure}[h!]
    \centering
    \includegraphics[width=1\columnwidth]{../figures/sot-embeddings.pdf}
    \caption{\textbf{SOT Embeddings.} Heatmap of SOT embeddings for \texttt{PN} (left) and \texttt{B} (right) on the \texttt{SP} dataset for a random test episode in 5-way 5-shot setting. Support and query samples from the same class are adjacent in the embedding matrix.}
    \label{fig:sot-embeddings}
\end{figure}

% Finding 3: SOT improves performance in interaction with LSTM re-embedding
Finally, the ablation study of the re-embedding components in \texttt{MN} suggests that the synergy between the SOT re-embeddings and the LSTM re-embeddings plays a crucial role in the observed performance enhancement of the metric-based meta-learner. The underlying mechanisms of this phenomenon, however, were beyond the purview of this project and are left for future work.


\section{Conclusion}

In summary, we demonstrate the effectiveness of both transfer and meta-learning algorithm for learning effective feature representations for classification tasks with limited data in the biomedical domain. 
Notably, employing the transductive SOT feature transform combined with meta-learners emerged as a highly effective approach, facilitating rapid and precise knowledge transfer even in the most challenging few-shot learning settings with many classes and only a single example per class. Especially the interaction with a subsequent embedding module that mixes information between the support and query set proved to be a crucial component for the success of the transductive SOT feature transform.

\newpage
\section{Appendix}

\begin{table}[h]
    \centering
    \caption{\textbf{Dataset Statistics.} The Table shows the number of samples and targets in each split of the two datasets.}
    \vspace{5pt}
    \label{tab:data}
    \begin{tabular}{lllll}
        \toprule
        & \textbf{Split} & \textbf{\#Samples} (\%) & \textbf{\#Targets} & \textbf{Overlap} \\
        \midrule
        \multirow{3}{*}{\texttt{TM}} 
        & Train. & 65,846 (62\%) & 59 & N/A \\
        & Val. & 15,031 (14\%) & 47 & 10 \\
        & Test. & 25,083 (24\%) & 37 & 4 \\
        \\
        \multirow{3}{*}{\texttt{SP}} 
        & Train. & 12,141 (85\%) & 636 & N/A \\
        & Val. & 1,407 (10\%) & 159 & 0 \\
        & Test. & 703 (5\%) & 89 & 0 \\
        \bottomrule
    \end{tabular}
\end{table}

\bibliography{main}
\bibliographystyle{cs502}

\end{document}

